%------------------------------------- preamble -------------------------------------
\documentclass{report}

\usepackage{graphicx}
\usepackage{hyperref}
\usepackage{listings}
\usepackage{hyperref}       % to create a linked table of contents
\usepackage{cite}


\graphicspath{ {images/} }
\lstdefinestyle{mystyle}{
    breakatwhitespace=false,
    breaklines=true,
    captionpos=b,
    keepspaces=true,
    numbers=left,
    numbersep=5pt,
    showspaces=false,
    showstringspaces=false,
    showtabs=false,
    tabsize=2
}
\lstset{style=mystyle}
\hypersetup{colorlinks,citecolor=black,filecolor=black,linkcolor=black,urlcolor=black}

%------------------------------------- end preamble -------------------------------------
\begin{document}

%-------------------- title -------------------------------------
\title{Sbpy Application Report}
\author{Qingyu Zhang}
\date{June 2021}
\maketitle

\tableofcontents

%--------------------end title --------------------------------

%------------------------------------ chapter 1 -------------------------------------
\chapter{Introduction}

\section{Sbpy}

Sbpy is an Astropy affiliated package for small-body planetary astronomy. It is meant to supplement functionality provided by Astropy with functions and methods that are frequently used in the context of planetary astronomy with a clear focus on asteroids and comets\cite{Mommert2019}.

\section{Module Structure}
Sbpy contains several sub-modules, each of which is designed to provide certain functionalities relevant to astronomy study. The figure below is from the Sbpy document website which shows the general design of the package.

\begin{figure}[htb]
    \centering
    \includegraphics[width=0.5\textwidth]{structure}
    \caption{the module structure of sbpy}
    \label{fig:structure}
\end{figure}

Modules are shown as rectangular boxes, the colored rounded boxes are supporting modules which are the basics of this package. They are all in the data module and support many of the computations and models in this package. on the right hand side, these modules are generally applications of the data module, and each of them is designed to solve problems in a certain filed of astronomy study. We can see their Dependencies relationships from those colored symbols, where colored symbols match the modules they are using with the same colors and symbols.

In our application, we will mainly use the \emph{activity} module to finish our computational tasks which also uses \emph{Ephem} and \emph{Orbit} modules as our data source.

%------------------------------------- end chapter ----------------------------------
%------------------------------------ chapter 2 -------------------------------------
\chapter{Computational Task}
\section{Task1 Orbit}
This task asks us to plot the orbit and location of asteroids, we need to obtain the orbit data and then compute its location. To do that I first use the data module to get the asteroids' orbit data and ephem data, then use class Ephem's method to compute the results. 
\subsection{Work Flow}
The first computational task is to compute the orbit of a planet and its position at designated time.
\begin{figure}[htb]
    \centering
    \includegraphics[width=0.8\textwidth]{task1}
    \caption{task2 workflow}
    \label{fig:task2}
\end{figure}

\subsection{User Input}
The application needs user to input the information of the comet or planet including target-id, target-type and so on. Users can get these information through \url{https://ssd.jpl.nasa.gov/horizons.cgi#top}
\subsection{Data Extraction}
Take the default input as an example, we want to plot the orbit of the \emph{46P}, \emph{Earth} and \emph{Jupiter}.
Part of the Orbit data and filed names are shown blow.
\begin{figure}[htb]
    \centering
    \includegraphics[width=0.5\textwidth]{DataExtraction}
    \caption{Orbit Data}
    \label{fig:OrbitData}
\end{figure}


\subsection{Computation and Visualization}
Code below demonstrates the core of the orbit plotting
\begin{lstlisting}[language=Python]
epochs = orb['epoch'] + np.linspace(-1, 1, 1000) * orb['P'] / 2
eph = Ephem.from_oo(orb, epochs=epochs, dynmodel='2')
ax.plot(eph['x'].value, eph['y'].value, **plot_kwargs)
\end{lstlisting}
After running the program we will get the orbit plot below.
From the outside to the inside, the orbits of Jupiter, comet, and Earth respectively.
\begin{figure}[htb]
    \centering
    \includegraphics[width=0.5\textwidth]{orbit}
    \caption{Orbit Plot}
    \label{fig:OrbitPlot}
\end{figure}

\section{Task2 Haser Model}
This task asks us to compute the distribution of coma molecules under photolysis, it's very physical and it's from one of the examples from the Sbpy tutorial, but it's easy to solve using the Haser model which has been packaged in the \emph{activity} module, so that I can use it to compute the result straight forward.

\subsection{Work Flow}
\begin{figure}[htb]
    \centering
    \includegraphics[width=0.7\textwidth]{task2.png}
    \caption{Haser model work flow}
    \label{fig:haser_workflow}
\end{figure}

\subsection{Computation}
Calculate the Haser model column density from 1 to $10^4$ km for CO$_2$, produced at a rate of $10^{28}$/s. Let the expansion velocity be 0.8 km/s.
Let the heliocentric distance range from $1au$ to $10^{11}au$.
Code blow shows the computation of the model.
\begin{lstlisting}[language=Python]
gamma = v * tau * (rh / u.au)**2
co2 = gas.Haser(Q, v, gamma)
\end{lstlisting}

Then save the result object using pickle.
\begin{lstlisting}[language=Python]
with open('haser_ds', 'wb') as f:
    pk.dump(res, f)
\end{lstlisting}

\subsection{Visualization}
First We read the result
\begin{lstlisting}[language=Python]
with open('haser_ds', 'rb') as f:
    res = pk.load(f)
rho = res['rho']
data = res['data']
\end{lstlisting}

Figure blow is the plot of the distribution of $CO_2$
\begin{figure}[htb]
    \centering
    \includegraphics[width=0.8\textwidth]{haser}
    \caption{Distribution of $CO_2$}
    \label{fig:haser}
\end{figure}

%------------------------------------- end chapter ----------------------------------

%------------------------------------ chapter 3 -------------------------------------
\chapter{Conclusion}
Sbpy is an useful package to assist researchers to handle observation data and manipulate them.
It also supply many modules other than data, which is quite difficult for me to understand their application. I used the data module and the activity module, it broadened my field of vision, I've never had a chance to touch professional Astronomy data, fresh but difficult.

Most Importantly, it helps me getting more familiar with those techniques and tools learned from this course.

%------------------------------------ end chapter -------------------------------------
\bibliographystyle{unsrt}
\bibliography{report-bib}

\end{document}